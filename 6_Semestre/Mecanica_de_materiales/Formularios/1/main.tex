\documentclass[letterpaper,12pt]{extarticle}

%% Paquetes
    \usepackage[utf8]{inputenc}         % Unicode support (Umlauts etc.)
    \usepackage{amsmath,mathtools}      % Advanced math typesetting
    \usepackage{amssymb}
    \usepackage{physics}                % Physics notation
    \usepackage{hyperref}               % Add a link to your document
    \usepackage{geometry}               % Margins
    \usepackage{multicol}               % for multicolumn layout
    \usepackage{xcolor}                 % For color
    \usepackage{tcolorbox}              % For colored boxes
    \usepackage{fancyhdr}               % Headers and footers

%---------------------------------------------------------------------%

\geometry{top=1in, headheight=0.5in, headsep=0.1in, margin=0.7in}

% Compact itemize
\usepackage{enumitem}
\setlist{nosep}

% Compact sections
\usepackage[compact]{titlesec}
\titlespacing{\section}{0pt}{*0}{*0}
\titlespacing{\subsection}{0pt}{*0}{*0}
\titlespacing{\subsubsection}{0pt}{*0}{*0}

% Reduce line spacing
\renewcommand{\baselinestretch}{0.8}

% Color scheme
\definecolor{color}{rgb}{0.694, 0.094, 0.153}

% Headers
\pagestyle{fancy}
\fancyhf{} % clear all header and footer fields
\renewcommand{\headrulewidth}{0pt} % no line in header area
\fancyhead[C]{Mecanica de materiales \par 1° Parcial} % Center

% New command for creating boxes with black text
\newcommand{\mybox}[2]
{
    \begin{tcolorbox}[colback=color!5!white,colframe=color!75!black,boxsep=1pt,arc=0pt,outer arc=0pt,title={\textcolor{white}{#1}}]
        \textcolor{black}{#2}
    \end{tcolorbox}
}

% Start the document
\begin{document}

    \fontsize{10pt}{11pt}\selectfont

    \begin{multicols}{2}

    \mybox{Condiciones de equilibrio}
    {
        Cuando un elemento esta en equilibrio, debe de cumplir con estas condiciones.
        \begin{subequations}
            \begin{align}
                +\circlearrowleft \sum F_x &= 0 \\
                +\uparrow \sum F_y &= 0 \\
                +\rightarrow \sum M_o &= 0
            \end{align}
        \end{subequations}
    }

    \mybox{Analisis de esfuerzos}
    {
        \textbf{Esfuerzo promedio normal:}
        \begin{equation}
            \sigma = \frac{P}{A}
        \end{equation}
        Donde: \space
        \begin{itemize}
            \item \( P \): Carga aplicada
            \item \( A \): Área de la sección transversal
        \end{itemize}
        \vspace{5mm}
        \textbf{Esfuerzo cortante simple:}
        \begin{equation}
            \tau = \frac{P}{A}
        \end{equation}
        Donde: \space
        \begin{itemize}
            \item \( P \): Carga aplicada
            \item \( A \): Área de la sección de corte
        \end{itemize}
        \vspace{5mm}
        \textbf{Esfuerzo máximo:}
        \begin{equation}
            \sigma_{max} = \frac{P}{A}
        \end{equation}
        Donde: \space
        \begin{itemize}
            \item \( P \): Carga aplicada
            \item \( A \): Mínima area de la sección transversal
        \end{itemize}
    }

    \end{multicols}

\end{document}