\documentclass[letterpaper,11pt]{extarticle}

%% Paquetes
    \usepackage[utf8]{inputenc}         % Unicode support (Umlauts etc.)
    \usepackage{amsmath,mathtools}      % Advanced math typesetting
    \usepackage{amssymb}
    \usepackage{physics}                % Physics notation
    \usepackage{hyperref}               % Add a link to your document
    \usepackage{geometry}               % Margins
    \usepackage{multicol}               % for multicolumn layout
    \usepackage{xcolor}                 % For color
    \usepackage{tcolorbox}              % For colored boxes
    \usepackage{fancyhdr}               % Headers and footers
    \usepackage{array}

%---------------------------------------------------------------------%

\geometry{top=1in, headheight=0.5in, headsep=0.1in, margin=0.5in}

% Compact itemize
\usepackage{enumitem}
\setlist{nosep}

% Compact sections
\usepackage[compact]{titlesec}
\titlespacing{\section}{0pt}{*0}{*0}
\titlespacing{\subsection}{0pt}{*0}{*0}
\titlespacing{\subsubsection}{0pt}{*0}{*0}

% Reduce line spacing
\renewcommand{\baselinestretch}{0.8}

% Color scheme
\definecolor{color}{rgb}{0.694, 0.094, 0.153}

% Headers
\pagestyle{fancy}
\fancyhf{} % clear all header and footer fields
\renewcommand{\headrulewidth}{0pt} % no line in header area
\fancyhead[L]{Mecánica de materiales} % Left
\fancyhead[R]{Joel Zuñiga} % Right

% New command for creating boxes with black text
\newcommand{\mybox}[2]
{
    \begin{tcolorbox}[colback=color!5!white,colframe=color!75!black,boxsep=1pt,arc=0pt,outer arc=0pt,title={\textcolor{white}{#1}}]
        \textcolor{black}{#2}
    \end{tcolorbox}
}

% Start the document
\begin{document}

    \renewcommand{\arraystretch}{1.5} % <--- Más espacio entre filas
    \fontsize{10pt}{11pt}\selectfont
    
    \begin{center} 
        \textbf{Análisis de elementos sujetos a carga axial}
    \end{center}

    \begin{multicols}{2}
        \mybox{Esfuerzo normal:}
        {
            Es generado cuando la viga está sometida a un momento flector, es decir, cuando actúa una fuerza transversal sobre ella. Se mide con las unidades de [$Pa$] ó [$psi$].
            \begin{equation}
                \epsilon = \frac{\triangle L}{L_0}
            \end{equation}
            \begin{equation}
                \sigma = \frac{P}{A} = E\epsilon
            \end{equation}
            Donde:
            \begin{center}
                \begin{tabular}{ c | p{35mm} | c }
                    \hline Variable & Definición & Unidad \\ \hline 
                    $\epsilon$ & Deformación unitaria & [ ]\\
                    $L$ & Longitud del elemento & [$m$] ó [$in$]\\
                    $P$ & Fuerza axial aplicada & [$N$] ó [$lb_f$]\\
                    $A$ & Área transversal a la fuerza & [$m^2$] ó [$in^2$]\\
                    $E$ & Módulo de elasticidad o módulo de Young & [$Pa$] ó [$psi$]\\ \hline
                \end{tabular}
            \end{center}
        }

        \mybox{Esfuerzo cortante:}
        {
            Es generado cuando hay una fuerza cortante en la sección transversal la viga. Se mide con las unidades de [$Pa$] ó [$psi$].\\\\
            \textbf{Cortante simple:}
            \begin{equation}
                \tau = \frac{P}{A}
            \end{equation}
            \textbf{Cortante doble:}
            \begin{equation}
                \tau = \frac{P}{2A}
            \end{equation}
            Donde:
            \begin{center}
                \begin{tabular}{ c | p{35mm} | c }
                    \hline Variable & Definición & Unidad \\ \hline 
                    $P$ & Fuerza perpendicular cortante & [$N$] ó [$lb_f$]\\
                    $A$ & Área de corte & [$m^2$] ó [$in^2$]\\ \hline
                \end{tabular}
            \end{center}
        }

        \mybox{Esfuerzo por aplastamiento:}
        {
            Es generado cuando una fuerza se transmite a través de superficies de contacto pequeñas, como en uniones con pernos o pasadores, donde el perno presiona el material. Se mide con las unidades de [$Pa$] ó [$psi$].
            \begin{equation}
                \sigma_B = \frac{P}{A}
            \end{equation}
            Donde:
            \begin{center}
                \begin{tabular}{ c | p{35mm} | c }
                    \hline Variable & Definición & Unidad \\ \hline 
                    $P$ & Fuerza perpendicular cortante & [$N$] ó [$lb_f$]\\
                    $A$ & Área proyectada del barreno & [$m^2$] ó [$in^2$]\\ \hline
                \end{tabular}
            \end{center}
        }
        
        \mybox{Deformación axial:}
        {
            Se genera cuando un elemento está sometido a una carga axial centrada, generando una deformación uniforme (extensión o compresión) a lo largo de su longitud. Se mide con las unidades de [$m$] ó [$in$].
            \begin{equation}
                \delta = \frac{PL}{AE}
            \end{equation}
            Donde:
            \begin{center}
                \begin{tabular}{ c | p{35mm} | c }
                    \hline Variable & Definición & Unidad \\ \hline 
                    $P$ & Fuerza axial aplicada & [$N$] ó [$lb_f$]\\
                    $L$ & Longitud del elemento & [$N$] ó [$lb_f$]\\
                    $A$ & Área transversal a la fuerza & [$m^2$] ó [$in^2$]\\
                    $E$ & Módulo de elasticidad o módulo de Young & [$Pa$] ó [$psi$]\\ \hline
                \end{tabular}
            \end{center}
        }

        \mybox{Modulo de Poisson:}
        {
            Se aplica cuando un material es sometido a una carga axial y, como resultado, además de extenderse o comprimirse en la dirección de la carga, también cambia de tamaño en la dirección perpendicular a la carga.
            \begin{equation}
                \epsilon_x = \frac{\sigma_x}{E} - \frac{\nu\sigma_y}{E} - \frac{\nu\sigma_z}{E}
            \end{equation}
            \begin{equation}
                \epsilon_y = -\frac{\sigma_x}{E} + \frac{\sigma_y}{E} - \frac{\nu\sigma_z}{E}
            \end{equation}
            \begin{equation}
                \epsilon_z = -\frac{\nu\sigma_x}{E} - \frac{\nu\sigma_y}{E} + \frac{\sigma_z}{E}
            \end{equation}
            Donde:
            \begin{center}
                \begin{tabular}{ c | p{35mm} | c }
                    \hline Variable & Definición & Unidad \\ \hline 
                    $\sigma$ & Esfuerzo aplicado & [$m^2$] ó [$in^2$]\\
                    $\nu$ & Modulo de Poisson & [ ]\\
                    $E$ & Módulo de elasticidad o módulo de Young & [$Pa$] ó [$psi$]\\ \hline
                \end{tabular}
            \end{center}
        }
    \end{multicols}
    
    \newpage

    \begin{center} 
        \textbf{Análisis de elementos sujetos a torsión} 
    \end{center}
    \begin{multicols}{2}
        \mybox{Relación de engranajes:}
        {
            \textbf{Análisis estatico:}
            \begin{equation}
                \frac{T_A}{r_A} = \frac{T_B}{r_B}
            \end{equation}
            \textbf{Análisis cinematico:}
            \begin{equation}
                r_A \omega_A = r_B \omega_B
            \end{equation}
            \begin{equation}
                r_A \phi_A = r_B \phi_B
            \end{equation}
            Donde: 
            \begin{center}
                \begin{tabular}{ c | p{28mm} | c }
                    \hline Variable & Definición & Unidad \\ \hline 
                    $T$ & Torque aplicado & [$N \cdot m$] ó [$lb_f \cdot in$]\\
                    $r$ & Radio de eje sometido a torsion & [$m$] ó [$in$]\\
                    $\omega$ & Velocidad angular  & [$\frac{rad}{s}$]\\
                    $\phi$ & Desplazamiento angular  & [$rad$]\\ \hline
                \end{tabular}
            \end{center}
        }
        \mybox{Momento polar de inercia ($J$):}
        {
            Es generado cuando una sección estructural está sometida a torsión (giro alrededor de su eje longitudinal). \\\\
            \textbf{Eje redondo sólido:}
            \begin{equation}
                J = \frac{\pi}{2}r^4 = \frac{\pi}{32}d^4
            \end{equation}
            \textbf{Eje redondo tubular:}
            \begin{equation}
                J = \frac{\pi}{2}\left(r_1^4 - r_2^4\right) = \frac{\pi}{32}\left(d_1^4 - d_2^4\right)
            \end{equation}
            Donde:
            \begin{center}
                \begin{tabular}{ c | p{35mm} | c }
                    \hline Variable & Definición & Unidad \\ \hline 
                    $r$ & Radio de eje sometido a torsion & [$m$] ó [$in$]\\
                    $d$ & Diametro de eje sometido a torsion & [$m$] ó [$in$]\\ \hline
                \end{tabular}
            \end{center}
        }
        
        \mybox{Esfuerzo cortante por torsión:}
        {
            Es generado cuando hay un momento de torsión $T$ aplicado sobre el eje longitudinal de la viga.
            \begin{equation}
                \tau = \frac{Tc}{J}
            \end{equation}
            Donde:
            \begin{center}
                \begin{tabular}{ c | p{25mm} | c }
                    \hline Variable & Definición & Unidad \\ \hline 
                    $T$ & Torque aplicado & [$N \cdot m$] ó [$lb_f \cdot in$]\\
                    $c$ & Distancia desde el eje neutro hasta la fibra extrema & [$m$] ó [$in$]\\
                    $J$ & Momento polar de inercia de la sección transversal & [$m^2$] ó [$in^2$]\\ \hline
                \end{tabular}
            \end{center}
        }

        \mybox{Deformacion por torsion:}
        {
            Es generado cuando la viga es sometida a un momento torsional, es decir, cuando se aplica una fuerza que genera un giro en torno al eje longitudinal de la viga. Se mide en [$rad$] ó [$^\circ$].
            \begin{equation}
                \phi = \frac{TL}{GJ}
            \end{equation}
            Donde:
            \begin{center}
                \begin{tabular}{ c | p{25mm} | c }
                    \hline Variable & Definición & Unidad \\ \hline 
                    $T$ & Torque aplicado & [$N \cdot m$] ó [$lb_f \cdot in$]\\
                    $L$ & Longitud de la viga & [$m$] ó [$in$]\\
                    $G$ & Módulo de rigidez del material $x$ & [$Pa$] ó [$psi$]\\
                    $J$ & Momento polar de inercia de la sección transversal & [$m^4$] ó [$in^4$]\\ \hline
                \end{tabular}
            \end{center}
        }
    \end{multicols}

    \newpage

    \begin{center} \textbf{Análisis de elementos sujetos a flexión} \end{center}
    \begin{multicols}{2}
        \mybox{Momento de inercia ($I$):}
        {
            Es generado cuando una sección estructural está sometida a flexión. Se mide en [$m^4$] ó [$in^4$] \\\\
            \textbf{Viga con forma rectangular:}
            \begin{equation}
                I_x = \frac{bh^3}{12}
            \end{equation}
            \begin{equation}
                I_y = \frac{hb^3}{12}
            \end{equation}
            \textbf{Viga con forma redonda:}
            \begin{equation}
                I_x = I_y = \frac{\pi}{4}r^4 = \frac{\pi}{64}d^4
            \end{equation}
            Donde:
            \begin{center}
                \begin{tabular}{ c | p{35mm} | c }
                    \hline Variable & Definición & Unidad \\ \hline 
                    $b$ & Longitud de la base del rectángulo & [$m$] ó [$in$]\\
                    $h$ & Longitud de la altura del rectángulo & [$m$] ó [$in$]\\
                    $r$ & Radio de eje sometido a flexión & [$m$] ó [$in$]\\
                    $d$ & Diámetro de eje sometido a flexión & [$m$] ó [$in$]\\ \hline
                \end{tabular}
            \end{center}
        }

        \mybox{Esfuerzo normal:}
        {
            Es generado cuando la viga está sometida a un momento flector, es decir, cuando actúa una fuerza transversal sobre ella. Se mide con las unidades de [$Pa$] ó [$psi$]. \\ 
            \begin{equation}
                \sigma = \frac{M(x)c}{I}
            \end{equation}
            Donde:
            \begin{center}
                \begin{tabular}{ c | p{25mm} | c }
                    \hline Variable & Definición & Unidad \\ \hline 
                    $M(x)$ & Ecuación de momento flector en sección $x$ & [$N \cdot m$] ó [$lb_f\cdot in$]\\
                    $c$ & Distancia desde el eje neutro hasta la fibra extrema & [$m$] ó [$in$]\\
                    $I$ & Momento de inercia & [$m^4$] ó [$in^4$]\\ \hline
                \end{tabular}
            \end{center}
        }

        \mybox{Esfuerzo cortante:}
        {
            Es generado cuando hay una fuerza cortante en la sección transversal la viga. Se mide con las unidades de [$Pa$] ó [$psi$]. \\\\
            \textbf{Viga con forma rectangular:}
            \begin{equation}
                \tau = \left(\frac{3}{2}\right)\left(\frac{V(x)}{A}\right)
            \end{equation}
            \textbf{Viga con forma redonda:}
            \begin{equation}
                \tau = \left(\frac{4}{3}\right)\left(\frac{V(x)}{A}\right)
            \end{equation}
            \textbf{Viga con forma redonda hueca:}
            \begin{equation}
                \tau = \frac{2V(x)}{A}
            \end{equation}
            Donde:
            \begin{center}
                \begin{tabular}{ c | p{35mm} | c }
                    \hline Variable & Definición & Unidad \\ \hline 
                    $V(x)$ & Ecuacion de fuerza cortante en sección $x$ & [$N$] ó [$lb_f$]\\
                    $A$ & Área transversal a la fuerza & [$m^2$] ó [$in^2$]\\ \hline
                \end{tabular}
            \end{center}
        }

        \mybox{Deformacion por flexión:}
        {
            Es generada cuando una viga es sometida a un momento flector, es decir, cuando una fuerza transversal actúa sobre la viga en un punto determinado. Se mide en [$m$] ó [$in$].
            \begin{equation}
                EI\frac{d^2y}{dx^2} = M(x)
            \end{equation}
            \begin{equation}
                \frac{dy}{dx} = \frac{1}{EI}\int M(x)dx + C_1
            \end{equation}
            \begin{equation}
                y(x) = \frac{1}{EI}\iint M(x)dxdx + C_1x + C_2
            \end{equation}
            Donde:
            \begin{center}
                \begin{tabular}{ c | p{25mm} | c }
                    \hline Variable & Definición & Unidad \\ \hline 
                    $E$ & Modulo de Young & [$Pa$] ó [$psi$]\\
                    $I$ & Momento de inercia & [$m^4$] ó [$in^4$]\\
                    $M(x)$ & Ecuación de momento flector en sección $x$ & [$N \cdot m$] ó [$lb_f \cdot in$]\\ \hline
                \end{tabular}
            \end{center}
        }
    \end{multicols}

    \newpage

    \begin{center} \textbf{Esfuerzos principales} \end{center}
    \begin{multicols}{2}
        \mybox{Esfuerzos principales:}
        {
            Se mide en [$m^4$] ó [$in^4$] \\\\
            \textbf{Viga con forma rectangular:}
            \begin{equation}
                I_x = \frac{bh^3}{12}
            \end{equation}
            \begin{equation}
                I_y = \frac{hb^3}{12}
            \end{equation}
            \textbf{Viga con forma redonda:}
            \begin{equation}
                I_x = I_y = \frac{\pi}{4}r^4 = \frac{\pi}{64}d^4
            \end{equation}
            Donde:
            \begin{center}
                \begin{tabular}{ c | p{35mm} | c }
                    \hline Variable & Definición & Unidad \\ \hline 
                    $b$ & Longitud de la base del rectángulo & [$m$] ó [$in$]\\
                    $h$ & Longitud de la altura del rectángulo & [$m$] ó [$in$]\\
                    $r$ & Radio de eje sometido a flexión & [$m$] ó [$in$]\\
                    $d$ & Diámetro de eje sometido a flexión & [$m$] ó [$in$]\\ \hline
                \end{tabular}
            \end{center}
        }

        \mybox{Esfuerzo normal:}
        {
            Es generado cuando la viga está sometida a un momento flector, es decir, cuando actúa una fuerza transversal sobre ella. Se mide con las unidades de [$Pa$] ó [$psi$]. \\ 
            \begin{equation}
                \sigma = \frac{M(x)c}{I}
            \end{equation}
            Donde:
            \begin{center}
                \begin{tabular}{ c | p{25mm} | c }
                    \hline Variable & Definición & Unidad \\ \hline 
                    $M(x)$ & Ecuación de momento flector en sección $x$ & [$N \cdot m$] ó [$lb_f\cdot in$]\\
                    $c$ & Distancia desde el eje neutro hasta la fibra extrema & [$m$] ó [$in$]\\
                    $I$ & Momento de inercia & [$m^4$] ó [$in^4$]\\ \hline
                \end{tabular}
            \end{center}
        }

        \mybox{Esfuerzo cortante:}
        {
            Es generado cuando hay una fuerza cortante en la sección transversal la viga. Se mide con las unidades de [$Pa$] ó [$psi$]. \\\\
            \textbf{Viga con forma rectangular:}
            \begin{equation}
                \tau = \left(\frac{3}{2}\right)\left(\frac{V(x)}{A}\right)
            \end{equation}
            \textbf{Viga con forma redonda:}
            \begin{equation}
                \tau = \left(\frac{4}{3}\right)\left(\frac{V(x)}{A}\right)
            \end{equation}
            \textbf{Viga con forma redonda hueca:}
            \begin{equation}
                \tau = \frac{2V(x)}{A}
            \end{equation}
            Donde:
            \begin{center}
                \begin{tabular}{ c | p{35mm} | c }
                    \hline Variable & Definición & Unidad \\ \hline 
                    $V(x)$ & Ecuacion de fuerza cortante en sección $x$ & [$N$] ó [$lb_f$]\\
                    $A$ & Área transversal a la fuerza & [$m^2$] ó [$in^2$]\\ \hline
                \end{tabular}
            \end{center}
        }

        \mybox{Deformacion por flexión:}
        {
            Es generada cuando una viga es sometida a un momento flector, es decir, cuando una fuerza transversal actúa sobre la viga en un punto determinado. Se mide en [$m$] ó [$in$].
            \begin{equation}
                EI\frac{d^2y}{dx^2} = M(x)
            \end{equation}
            \begin{equation}
                \frac{dy}{dx} = \frac{1}{EI}\int M(x)dx + C_1
            \end{equation}
            \begin{equation}
                y(x) = \frac{1}{EI}\iint M(x)dxdx + C_1x + C_2
            \end{equation}
            Donde:
            \begin{center}
                \begin{tabular}{ c | p{25mm} | c }
                    \hline Variable & Definición & Unidad \\ \hline 
                    $E$ & Modulo de Young & [$Pa$] ó [$psi$]\\
                    $I$ & Momento de inercia & [$m^4$] ó [$in^4$]\\
                    $M(x)$ & Ecuación de momento flector en sección $x$ & [$N \cdot m$] ó [$lb_f \cdot in$]\\ \hline
                \end{tabular}
            \end{center}
        }
    \end{multicols}
\end{document}